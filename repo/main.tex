% Credits are indicated where needed. The general idea is based on a template by Vel (vel@LaTeXTemplates.com) and Frits Wenneker.

\documentclass[11pt, a4paper]{article} % General settings in the beginning (defines the document class of your paper)
% 11pt = is the font size
% A4 is the paper size
% “article” is your document class

%----------------------------------------------------------------------------------------
%	Packages
%----------------------------------------------------------------------------------------

% Necessary
\usepackage[german,english]{babel} % English and German language 
\usepackage{booktabs} % Horizontal rules in tables 
% For generating tables, use “LaTeX” online generator (https://www.tablesgenerator.com)
\usepackage{comment} % Necessary to comment several paragraphs at once
\usepackage[utf8]{inputenc} % Required for international characters
\usepackage[T1]{fontenc} % Required for output font encoding for international characters
\usepackage{listings}

\lstset{basicstyle=\linespread{0.8}\ttfamily\footnotesize,
    xleftmargin=0.7cm,
    frame=tlbr, framesep=0.2cm, framerule=0pt,
}
\lstset{frame=single}

% Might be helpful
\usepackage{amsmath,amsfonts,amsthm} % Math packages which might be useful for equations
\usepackage{tikz} % For tikz figures (to draw arrow diagrams, see a guide how to use them)
\usepackage{tikz-cd}
\usetikzlibrary{positioning,arrows} % Adding libraries for arrows
\usetikzlibrary{decorations.pathreplacing} % Adding libraries for decorations and paths
\usepackage{tikzsymbols} % For amazing symbols ;) https://mirror.hmc.edu/ctan/graphics/pgf/contrib/tikzsymbols/tikzsymbols.pdf 
\usepackage{blindtext} % To add some blind text in your paper

\usepackage{url}
\usepackage{enumerate}
%---------------------------------------------------------------------------------
% Additional settings
%---------------------------------------------------------------------------------

%---------------------------------------------------------------------------------
% Define your margins
\usepackage{geometry} % Necessary package for defining margins

\geometry{
	top=2cm, % Defines top margin
	bottom=2cm, % Defines bottom margin
	left=2.2cm, % Defines left margin
	right=2.2cm, % Defines right margin
	includehead, % Includes space for a header
	%includefoot, % Includes space for a footer
	%showframe, % Uncomment if you want to show how it looks on the page 
}

\setlength{\parindent}{15pt} % Adjust to set you indent globally 

%---------------------------------------------------------------------------------
% Define your spacing
\usepackage{setspace} % Required for spacing
% Two options:
\linespread{1}
%\onehalfspacing % one-half-spacing linespread

%----------------------------------------------------------------------------------------
% Define your fonts
\usepackage[T1]{fontenc} % Output font encoding for international characters
\usepackage[utf8]{inputenc} % Required for inputting international characters

\usepackage{XCharter} % Use the XCharter font


%---------------------------------------------------------------------------------
% Define your headers and footers

\usepackage{fancyhdr} % Package is needed to define header and footer
\pagestyle{fancy} % Allows you to customize the headers and footers

%\renewcommand{\sectionmark}[1]{\markboth{#1}{}} % Removes the section number from the header when \leftmark is used

% Headers
\lhead{} % Define left header
\chead{\textit{}} % Define center header - e.g. add your paper title
\rhead{} % Define right header

% Footers
\lfoot{} % Define left footer
\cfoot{\footnotesize \thepage} % Define center footer
\rfoot{ } % Define right footer

%---------------------------------------------------------------------------------
%	Add information on bibliography
\usepackage{natbib} % Use natbib for citing
\usepackage{har2nat} % Allows to use harvard package with natbib https://mirror.reismil.ch/CTAN/macros/latex/contrib/har2nat/har2nat.pdf

% For citing with natbib, you may want to use this reference sheet: 
% http://merkel.texture.rocks/Latex/natbib.php

%---------------------------------------------------------------------------------
% Add field for signature (Reference: https://tex.stackexchange.com/questions/35942/how-to-create-a-signature-date-page)
\newcommand{\signature}[2][5cm]{%
  \begin{tabular}{@{}p{#1}@{}}
    #2 \\[2\normalbaselineskip] \hrule \\[0pt]
    {\small \textit{Signature}} \\[2\normalbaselineskip] \hrule \\[0pt]
    {\small \textit{Place, Date}}
  \end{tabular}
}
%---------------------------------------------------------------------------------
%	General information
%---------------------------------------------------------------------------------
\title{grepo - Repository group support} % Adds your title
\author{Srini, rs@toprllc.com}
\date{\small \today} % Adds the current date to your “cover” page; leave empty if you do not want to add a date


%---------------------------------------------------------------------------------
%	Define what’s in your document
%---------------------------------------------------------------------------------

\begin{document}
%\newfontfamily\listingsfont[Scale=.7]{Menlo}

% If you want a cover page, uncomment "%---------------------------------------------------------------------------------
% Cover page
%---------------------------------------------------------------------------------

% Here are more templates for other cover pages: https://www.latextemplates.com/cat/title-pages

% This example is based on this cover page example: https://www.latextemplates.com/template/academic-title-page

\begin{titlepage} % Starts new environment where the page number is not displayed and the count starts at 1 for the next page

%------------------------------------------------
%	Institutional information
%------------------------------------------------
	
\begin{minipage}{0.4\textwidth} % Begins new environment (like a text box)
    \begin{flushleft} % Sets environment on the left side of the paper
    \large
    University of XX\\ % Add your institution
    Chair of Political Science IV\\ % Add the chair
    Fall 2018\\ % Add term
    COURSE TITLE\\ % Add course title
    Supervisor: NAME % Add instructor/supervisor name 
    \end{flushleft}
\end{minipage}
	
\vspace*{2in} % Adds some space in-between
	
\center % Centre everything on the page

%------------------------------------------------
%	Main part
%------------------------------------------------
	
{\huge\bfseries TITLE OF YOUR PAPER}\\[0.4cm] % Add your paper title 
{\large\today}\\[0.4cm] % Add date (current day)
FIRSTNAME LASTNAME % Add your name
	
\vfill % Adds additional space

%------------------------------------------------
%	General information about the author
%------------------------------------------------

\vfill % Adds additional space

Your contact info \\ % Add your contact info
Your Program \\ % Add info about your program
Semester you are enrolled \\ % Add info about your semester

\vfill % Adds additional space

%------------------------------------------------
%	Word count
%------------------------------------------------

\vfill % Adds additional space
	
Word count: XXXX % To indicate the word count
% How to check words in a LaTeX document: https://www.overleaf.com/help/85-is-there-a-way-to-run-a-word-count-that-doesnt-include-latex-commands
	

	
\end{titlepage}" and uncomment "\begin{comment}" and "\end{comment}" to comment the following lines
%%---------------------------------------------------------------------------------
% Cover page
%---------------------------------------------------------------------------------

% Here are more templates for other cover pages: https://www.latextemplates.com/cat/title-pages

% This example is based on this cover page example: https://www.latextemplates.com/template/academic-title-page

\begin{titlepage} % Starts new environment where the page number is not displayed and the count starts at 1 for the next page

%------------------------------------------------
%	Institutional information
%------------------------------------------------
	
\begin{minipage}{0.4\textwidth} % Begins new environment (like a text box)
    \begin{flushleft} % Sets environment on the left side of the paper
    \large
    University of XX\\ % Add your institution
    Chair of Political Science IV\\ % Add the chair
    Fall 2018\\ % Add term
    COURSE TITLE\\ % Add course title
    Supervisor: NAME % Add instructor/supervisor name 
    \end{flushleft}
\end{minipage}
	
\vspace*{2in} % Adds some space in-between
	
\center % Centre everything on the page

%------------------------------------------------
%	Main part
%------------------------------------------------
	
{\huge\bfseries TITLE OF YOUR PAPER}\\[0.4cm] % Add your paper title 
{\large\today}\\[0.4cm] % Add date (current day)
FIRSTNAME LASTNAME % Add your name
	
\vfill % Adds additional space

%------------------------------------------------
%	General information about the author
%------------------------------------------------

\vfill % Adds additional space

Your contact info \\ % Add your contact info
Your Program \\ % Add info about your program
Semester you are enrolled \\ % Add info about your semester

\vfill % Adds additional space

%------------------------------------------------
%	Word count
%------------------------------------------------

\vfill % Adds additional space
	
Word count: XXXX % To indicate the word count
% How to check words in a LaTeX document: https://www.overleaf.com/help/85-is-there-a-way-to-run-a-word-count-that-doesnt-include-latex-commands
	

	
\end{titlepage}

%\begin{comment}
\maketitle % Print your title, author name and date; comment if you want a cover page 
%\end{comment}

%----------------------------------------------------------------------------------------
% Introduction
%----------------------------------------------------------------------------------------
\setcounter{page}{1} % Sets counter of page to 1

\section{Introduction} % Add a section title

Projects with many components and developers are best managed with distinct independent source code repositories. At a high level, there are
the public repositories developed outside the team which are just used; the components that are unique to the project then may be viewed as 
private.

The public repositories then are primarily replicated for usage, prepared (eg. built) and periodically refreshed. The private repositories on the other hand go through an active development cycle. In git parlance, in addition to normal software engineering activities would include creating new branches, merging, tagging and of course publishing. Specific sub projects thus may well span several repositories.

This tool enables developers manage their \textbf{projectlets} as a group. For example create a new \textbf{feature branch} on a set of related repositories and finally push or publish the branch as a group once the developer is ready to share.

\url{https://source.android.com/setup/develop/repo} is such a tool which serves as a pattern for this projectlet. The goals are simplified somewhat:

\begin{itemize}
    \item \textbf{git} is the only source code control system supported.
    \item Publishing updates, tags etc are to the original source repositories. The complete link to the repositories and the histories is maintained. The individual repositories may be manipulated individually.
    \item Commands mirror their \textbf{git} equivalents somewhat. The command semantics are designed for convenience and not for power. If necessary git can be directly accessed for more complex operations.
    \item Project configuration is provided in \textbf{yaml} format.
\end{itemize}

\subsection{Typical use cases}

Developers of embedded systems using e.g. \textbf{yocto}  have to manage their environment as outlined above. Their own application might be split into several repositories while depending heavily on the platform libraries spread over numerous other repositories. Both these groups are evolving - the public ones presumably changing slower than the private ones.

\paragraph{} Even simpler, many hobbyists have quite a few projectlets going in parallel, related somewhat loosely. In my own case my repos are spread between gitlab, github, bitbucket and so on. I also jump around from a Windows laptop, a Macbook and quite a few Linux servers not to mention several Raspberry PI's and I want to have most of my projects available in each of my systems. This tool allows me to setup my environment with minimal fuss.

\section{Specifications}

\subsection{Configuration}

A sample configuration for a project file is listed below.

\begin{lstlisting}[caption=Example Project.yaml]
public:
  workarea: "$HOME/grepotest/pub"
  projects:
    - repourl: "git@gitlab.com:projtemplates/go.git"
      reference: "master"
      path: "go/lib"
      build: "make setup all"

private:
  workarea: "$HOME/grepotest/pvt"
  server: "git@gitlab.com:RajaSrinivasan/"

  projects:
    - repo: "random.git"
      path: "random"

    - repo:  "exec.git"
      path: "exec"
      build: "build.sh"

    - repo: "codex.git"
      path: "codex"
      reference: "v0.1.0-B"
      build: "make setup all test"

    - repourl: "https://github.com/RajaSrinivasan/disign.git"
      path: "disign"
      build: "make all"

    - repourl: "https://github.com/RajaSrinivasan/spm.git"
      path: "spm"
      build: "make all"

    - repourl: "https://github.com/RajaSrinivasan/srctrace.git"
      path: "srctrace"
      reference: "feat_many_platforms"
      build: "make setup all"

    - repourl: "https://github.com/RajaSrinivasan/repotrace.git"
      path: "repotrace"

    - repourl: "https://github.com/RajaSrinivasan/rollpwd.git"
      path: "rollpwd"

\end{lstlisting}

\paragraph{Project Groups} The sample configuration lists two project groups: public and private. \textbf{public} projects are considered reference only and thus cloned and then detached from the HEAD. The repositories are considered immutable by this tool and thus push, tag and other operations will not be applied to the public projects.

\paragraph{Workarea} Each project group requires a work area to be provided. This is the top level of the directory structure where the projects will be cloned. Each project's path specification is appended to the work area to determine the workarea for the project. For the project codex in the above example, the work area will be the folder \texttt{codex} under the work area \textbf{\$HOME/grepo/pvt}. Please note that the path for each project has to be unique.

The work area specified is first translated using the environment variables.

\paragraph{}
\fbox{%
    \parbox{40em}{%
    Workspace directories should not fall within another git tree to avoid confusions.
    }%
}

\paragraph{Server} The project group can have a default server specified. This will be used for each project unless overridden in the project. In the example all the private projects come from gitlab.com and accessed using the git protocol \textbf{git@gitlab.com:RajaSrinivasan/}. 

\paragraph{Repository spec} A project can provide one of \textbf{repo} which is combined with the Server of the project group. Alternatively a \textbf{repourl} can be specified which becomes the full specification of the repository. In the above example, for \textbf{srctrace} the entire url is provided whereas for \textbf{codex} the url is formed using the Server spec and the repo spec resulting in: \textbf{git@gitlab.com:RajaSrinivasan/codex.git}.

\paragraph{Reference} Reference is a branch name or tag. The reference specification of the project group will be applied to each project unless overridden in the project. For a project, the reference could also be a commit id.

\paragraph{Build instruction} A project can provide a build instruction. Upon completion of a clone or a pull, this command expected to be a shell command is executed. As an illustration, for the project codex in the example, a make command is specified.
%----------------------------------------------------------------------------------------
% Literature review
%----------------------------------------------------------------------------------------

\section{Usage and examples}

\subsection{usage}

\begin{lstlisting}[caption=Basic usage]
$ bin/grepo --help

	grepo supports a project that comprises different repositories.

Usage:
  grepo [command]

Available Commands:
  diff        Diff from where we started
  help        Help about any command
  init        Initialize - setup the workspace
  pull        Pull for each repo
  push        Push for each repo
  status      Project Status
  tag         Tag each repo
  version     Report the version of the application

Flags:
      --config string   config file. (default "Project.yaml")
  -h, --help            help for grepo
      --verbose         be verbose
  -v, --version         version for grepo

Use "grepo [command] --help" for more information about a command.

\end{lstlisting}

\paragraph{init} The init command sets up the directory structure, clones the repository and performs the initial build of the repository. Optionally a branch name can be provided which creates the same branch on all the private repos.

\paragraph{pull} After the initial setup, the pull command applies \textbf{git pull} to each of the projectlets followed by an optional build. The public repos can also be pull'ed and optionally a build performed.

\paragraph{tag} Applies the tag to each (private only) repository. While pushing, the tags will be pushed as well.

\paragraph{diff} For each private repository, this displays the difference from the original reference. 

\paragraph{push} For each private repository, this performs a commit and a push. The same commit message is applied to all the repositories.  If a branch name had been specified with the init command, the push target is the newly created branch name. This utility applies a \textbf{commit -a} to each repo and thus can take care of the files known to git ie previously under git control. New files created should be added manually. Any tags created can also be pushed to the remote.

\begin{lstlisting}[caption=Project Initialization. Sample log]
bin/grepo init --verbose --force
2020/05/30 06:56:39 Loading the config file Project.yaml
2020/05/30 06:56:39 Project group: Public
2020/05/30 06:56:39 Workspace /Users/rajasrinivasan/grepotest/pub
2020/05/30 06:56:39 Project group: Private
2020/05/30 06:56:39 Workspace /Users/rajasrinivasan/grepotest/pvt
...
2020/05/30 06:56:41 Executing /usr/bin/git clone git@gitlab.com:RajaSrinivasan/codex.git .
2020/05/30 06:56:56 Cloning into '.'...
2020/05/30 06:56:56 Executing /usr/bin/git checkout v0.1.0-B
2020/05/30 06:56:56 Note: checking out 'v0.1.0-B'.

You are in 'detached HEAD' state. You can look around, make experimental
changes and commit them, and you can discard any commits you make in this
state without impacting any branches by performing another checkout.

If you want to create a new branch to retain commits you create, you may
do so (now or later) by using -b with the checkout command again. Example:

  git checkout -b <new-branch-name>

HEAD is now at 0a26e9b Deleted bin/app
...
2020/05/30 06:56:59 Created /Users/rajasrinivasan/grepotest/pvt/srctrace
2020/05/30 06:56:59 Executing /usr/bin/git clone https://github.com/RajaSrinivasan/srctrace.git .
2020/05/30 06:57:00 Cloning into '.'...
2020/05/30 06:57:00 Executing /usr/bin/git checkout feat_many_platforms
2020/05/30 06:57:00 Switched to a new branch 'feat_many_platforms'
Branch 'feat_many_platforms' set up to track remote branch 'feat_many_platforms' from 'origin'.
\end{lstlisting}

\begin{lstlisting}[caption= Example of status]
$ bin/grepo status
2020/05/30 17:27:03 Status in /Users/rajasrinivasan/grepotest/pvt/random
2020/05/30 17:27:03 On branch master
Your branch is up to date with 'origin/master'.

nothing to commit, working tree clean
2020/05/30 17:27:03 Status in /Users/rajasrinivasan/grepotest/pvt/exec
2020/05/30 17:27:03 On branch master
Your branch is up to date with 'origin/master'.
\end{lstlisting}

\section{Implementation}
\fbox{%
    \parbox{40em}{%
    Implementation in \textbf{go} : \url{https://github.com/RajaSrinivasan/grepo.git}
    }%
}

\end{document}