% Credits are indicated where needed. The general idea is based on a template by Vel (vel@LaTeXTemplates.com) and Frits Wenneker.

\documentclass[12pt, a4paper]{article} % General settings in the beginning (defines the document class of your paper)
% 11pt = is the font size
% A4 is the paper size
% “article” is your document class

%----------------------------------------------------------------------------------------
%	Packages
%----------------------------------------------------------------------------------------

% Necessary
\usepackage[german,english]{babel} % English and German language 
\usepackage{booktabs} % Horizontal rules in tables 
% For generating tables, use “LaTeX” online generator (https://www.tablesgenerator.com)
\usepackage{comment} % Necessary to comment several paragraphs at once
\usepackage[utf8]{inputenc} % Required for international characters
\usepackage[T1]{fontenc} % Required for output font encoding for international characters
\usepackage{listings}

\lstset{basicstyle=\linespread{0.8}\ttfamily\footnotesize,
    xleftmargin=0.7cm,
    frame=tlbr, framesep=0.2cm, framerule=0pt,
}
\lstset{frame=single}

% Might be helpful
\usepackage{amsmath,amsfonts,amsthm} % Math packages which might be useful for equations
\usepackage{tikz} % For tikz figures (to draw arrow diagrams, see a guide how to use them)
\usepackage{tikz-cd}
\usetikzlibrary{positioning,arrows} % Adding libraries for arrows
\usetikzlibrary{decorations.pathreplacing} % Adding libraries for decorations and paths
\usepackage{tikzsymbols} % For amazing symbols ;) https://mirror.hmc.edu/ctan/graphics/pgf/contrib/tikzsymbols/tikzsymbols.pdf 
\usepackage{blindtext} % To add some blind text in your paper

\usepackage{url}
\usepackage{enumerate}
%---------------------------------------------------------------------------------
% Additional settings
%---------------------------------------------------------------------------------

%---------------------------------------------------------------------------------
% Define your margins
\usepackage{geometry} % Necessary package for defining margins

\geometry{
	top=4cm, % Defines top margin
	bottom=2cm, % Defines bottom margin
	left=2.2cm, % Defines left margin
	right=2.2cm, % Defines right margin
	includehead, % Includes space for a header
	%includefoot, % Includes space for a footer
	%showframe, % Uncomment if you want to show how it looks on the page 
}

\setlength{\parindent}{15pt} % Adjust to set you indent globally 

%---------------------------------------------------------------------------------
% Define your spacing
\usepackage{setspace} % Required for spacing
% Two options:
\linespread{1.5}
%\onehalfspacing % one-half-spacing linespread

%----------------------------------------------------------------------------------------
% Define your fonts
\usepackage[T1]{fontenc} % Output font encoding for international characters
\usepackage[utf8]{inputenc} % Required for inputting international characters

\usepackage{XCharter} % Use the XCharter font

\usepackage{lscape}
%---------------------------------------------------------------------------------
% Define your headers and footers

\usepackage{fancyhdr} % Package is needed to define header and footer
\pagestyle{fancy} % Allows you to customize the headers and footers

%\renewcommand{\sectionmark}[1]{\markboth{#1}{}} % Removes the section number from the header when \leftmark is used

% Headers
\lhead{} % Define left header
\chead{\textit{}} % Define center header - e.g. add your paper title
\rhead{} % Define right header

% Footers
\lfoot{templates v0.1} % Define left footer
\rfoot{\footnotesize Page \thepage} % Define center footer
\cfoot{ } % Define right footer

%---------------------------------------------------------------------------------
%	Add information on bibliography
\usepackage{natbib} % Use natbib for citing
\usepackage{har2nat} % Allows to use harvard package with natbib https://mirror.reismil.ch/CTAN/macros/latex/contrib/har2nat/har2nat.pdf

% For citing with natbib, you may want to use this reference sheet: 
% http://merkel.texture.rocks/Latex/natbib.php

%---------------------------------------------------------------------------------
% Add field for signature (Reference: https://tex.stackexchange.com/questions/35942/how-to-create-a-signature-date-page)
\newcommand{\signature}[2][5cm]{%
  \begin{tabular}{@{}p{#1}@{}}
    #2 \\[2\normalbaselineskip] \hrule \\[0pt]
    {\small \textit{Signature}} \\[2\normalbaselineskip] \hrule \\[0pt]
    {\small \textit{Place, Date}}
  \end{tabular}
}
%---------------------------------------------------------------------------------
%	General information
%---------------------------------------------------------------------------------
\title{templates - a tool to improve productivity} % Adds your title
\author{Srini, rs@toprllc.com}
\date{\small \today} 
%---------------------------------------------------------------------------------
%	Define what’s in your document
%---------------------------------------------------------------------------------

\begin{document}
%\newfontfamily\listingsfont[Scale=.7]{Menlo}

% If you want a cover page, uncomment "%---------------------------------------------------------------------------------
% Cover page
%---------------------------------------------------------------------------------

% Here are more templates for other cover pages: https://www.latextemplates.com/cat/title-pages

% This example is based on this cover page example: https://www.latextemplates.com/template/academic-title-page

\begin{titlepage} % Starts new environment where the page number is not displayed and the count starts at 1 for the next page

%------------------------------------------------
%	Institutional information
%------------------------------------------------
	
\begin{minipage}{0.4\textwidth} % Begins new environment (like a text box)
    \begin{flushleft} % Sets environment on the left side of the paper
    \large
    University of XX\\ % Add your institution
    Chair of Political Science IV\\ % Add the chair
    Fall 2018\\ % Add term
    COURSE TITLE\\ % Add course title
    Supervisor: NAME % Add instructor/supervisor name 
    \end{flushleft}
\end{minipage}
	
\vspace*{2in} % Adds some space in-between
	
\center % Centre everything on the page

%------------------------------------------------
%	Main part
%------------------------------------------------
	
{\huge\bfseries TITLE OF YOUR PAPER}\\[0.4cm] % Add your paper title 
{\large\today}\\[0.4cm] % Add date (current day)
FIRSTNAME LASTNAME % Add your name
	
\vfill % Adds additional space

%------------------------------------------------
%	General information about the author
%------------------------------------------------

\vfill % Adds additional space

Your contact info \\ % Add your contact info
Your Program \\ % Add info about your program
Semester you are enrolled \\ % Add info about your semester

\vfill % Adds additional space

%------------------------------------------------
%	Word count
%------------------------------------------------

\vfill % Adds additional space
	
Word count: XXXX % To indicate the word count
% How to check words in a LaTeX document: https://www.overleaf.com/help/85-is-there-a-way-to-run-a-word-count-that-doesnt-include-latex-commands
	

	
\end{titlepage}" and uncomment "\begin{comment}" and "\end{comment}" to comment the following lines
%%---------------------------------------------------------------------------------
% Cover page
%---------------------------------------------------------------------------------

% Here are more templates for other cover pages: https://www.latextemplates.com/cat/title-pages

% This example is based on this cover page example: https://www.latextemplates.com/template/academic-title-page

\begin{titlepage} % Starts new environment where the page number is not displayed and the count starts at 1 for the next page

%------------------------------------------------
%	Institutional information
%------------------------------------------------
	
\begin{minipage}{0.4\textwidth} % Begins new environment (like a text box)
    \begin{flushleft} % Sets environment on the left side of the paper
    \large
    University of XX\\ % Add your institution
    Chair of Political Science IV\\ % Add the chair
    Fall 2018\\ % Add term
    COURSE TITLE\\ % Add course title
    Supervisor: NAME % Add instructor/supervisor name 
    \end{flushleft}
\end{minipage}
	
\vspace*{2in} % Adds some space in-between
	
\center % Centre everything on the page

%------------------------------------------------
%	Main part
%------------------------------------------------
	
{\huge\bfseries TITLE OF YOUR PAPER}\\[0.4cm] % Add your paper title 
{\large\today}\\[0.4cm] % Add date (current day)
FIRSTNAME LASTNAME % Add your name
	
\vfill % Adds additional space

%------------------------------------------------
%	General information about the author
%------------------------------------------------

\vfill % Adds additional space

Your contact info \\ % Add your contact info
Your Program \\ % Add info about your program
Semester you are enrolled \\ % Add info about your semester

\vfill % Adds additional space

%------------------------------------------------
%	Word count
%------------------------------------------------

\vfill % Adds additional space
	
Word count: XXXX % To indicate the word count
% How to check words in a LaTeX document: https://www.overleaf.com/help/85-is-there-a-way-to-run-a-word-count-that-doesnt-include-latex-commands
	

	
\end{titlepage}

%\begin{comment}
\maketitle % Print your title, author name and date; comment if you want a cover page 
%\end{comment}

%----------------------------------------------------------------------------------------
% Introduction
%----------------------------------------------------------------------------------------
\setcounter{page}{1} % Sets counter of page to 1

\section{Introduction} 

There are no shortcuts to gain proficiency in any discipline - particularly programming languages - except to practise a lot. Over the course of a self directed program, you will find yourself adapting a pattern and try to apply the pattern over and over again as you explore different facets. For example, a command line interface with switches and options might be a technique to explore ideas in encryption, compression, text processing and so on. A template of a \textbf{cli} program could be a great starting point for each project.

\paragraph{}
The ubiquitous web development framework \textbf{Node.js} \url{https://nodejs.org/en/about/} for example  supports a number of tools for the beginner to setup an initial project based on a template. With very little effort, a new project can be setup to help build your application rapidly.

\paragraph{}
In this projectlet, we specify a cli template and a web server template. They are designed to impose a standard on such efforts in addition to providing a starting point for new projects. 

\section{Command Language Template}

\paragraph{Expected features of cli applications}

The following is a list of features that would be useful for potential users.

\begin{itemize}
    \item help. this should print out the usage ie all the options and arguments. In most cases invoking the utility without any arguments should provide a help output and exit.
    \item version reporting - to enable issue reporting and tracking. Wherever feasible, a program can also provide a way to track the software back to a configuration management system.
    \item verbosity - either a simple boolean or a verbosity level that requests diagnostic information.
    \item No passwords to be specified on the command line. When there is a need for passwords an alternate way to specify such as environment variable could be a good solution.
\end{itemize}

\paragraph{Design guidelines}

\begin{itemize}
    \item Separation of the implementation from the invocation by a command line. Perhaps there may be a different way the service is deployed - possibly with a GUI or behind a web server.
    \item Judicious use of logging will enhance the ease of use.
    \item A design that can be adapted to a new requirement. Preferably a simple set of commands - in a script should be able to create a new project with the template as the starting point.
    \item It should be functional. Though minimal in what it supports, the template should be buildable and perform the simple functions listed above.
\end{itemize}

\subsection{Example}
The example in the repository below can be built and is functional.
\begin{lstlisting}
$ ./template

        This is a template project for command line utilities

Usage:
  template [command]

Available Commands:
  help        Help about any command
  version     Report the version of the application

Flags:
      --config string   config file (default is $HOME/.cli/cli.yaml)
  -h, --help            help for template
      --verbose         be verbose
  -v, --version         version for template

Use "template [command] --help" for more information about a command.

\end{lstlisting}

\section{Web server template}

\paragraph{Essential Features}
One example of deriving an app from the cli template is a web server which itself is a template. In this case, the requirements are:

\begin{itemize}
    \item Provide some set of services using a standard protocol such as \textbf{https} implying transport layer security.
    \item Compatible with web browsers found on standard platforms
    \item Support the notion of an admin user with support for administrative functions and a normal user with application functions. Before providing any service, the server shall require a login.
    \item Support a basic user database with usernames and passwords. While the database itself may be a simple file, the passwords should not be stored in cleartext.
    \item Ability to support development without a production oriented environment. This may call for self signed certificates, operate under different platforms.
    \item Basic setup for serving static html content.
\end{itemize}




\subsection{Minimal Realization based on the cli template}

The repository contains a template of a https server to meet the above. As the example illustrates, it can be \textbf{installed} when a self signed certificate is generated to support the \textbf{TLS} requirement. The installation process also generates a default user database with a minimal set of usernames. The tool prompts for the default passwords.

\subsection{Usage}

\begin{lstlisting}
$ ./main --help

        Generic TLS server

Usage:
  server [flags]
  server [command]

Available Commands:
  help        Help about any command
  install     Installs the server
  show        Show details
  user        User admin
  version     Report the version of the application

Flags:
      --config string   config file name. Default is
  -h, --help            help for server
  -v, --verbose int     verbosity level 1 .. 16
      --version         version for server

Use "server [command] --help" for more information about a command.

\end{lstlisting}
\begin{landscape}
\subsection{Installation Example}
\begin{lstlisting}
$ ./main install
Server directory[C:\msys64\home\RajaS/server] :
Server Port[443] :
Server URL[localhost] :
Admin Password (username: admin)[admin] :
User Password (username: user)[user] :
2020/07/25 07:21:53 Saved Configuration file C:\msys64\home\RajaS/server/etc/server.yaml
2020/07/25 07:21:53 Generating certificates
2020/07/25 07:21:55 Created private key file C:\msys64\home\RajaS/server/etc/keypair.pvt.pem
2020/07/25 07:21:55 Created public key C:\msys64\home\RajaS/server/etc/keypair.pub.pem
2020/07/25 07:21:55 Creating Cert
2020/07/25 07:21:55 Loaded C:\msys64\home\RajaS/server/etc/keypair.pvt.pem and created a private key
2020/07/25 07:21:55 Certificate saved to C:\msys64\home\RajaS/server/etc/certfile
2020/07/25 07:21:55 Setting up website
2020/07/25 07:21:55 Installing files of templates to C:\msys64\home\RajaS/server/html
2020/07/25 07:21:55 Installed footer.html in C:\msys64\home\RajaS/server/html
2020/07/25 07:21:55 Installed header.html in C:\msys64\home\RajaS/server/html
2020/07/25 07:21:55 Installed index.html in C:\msys64\home\RajaS/server/html
2020/07/25 07:21:55 Installed login.html in C:\msys64\home\RajaS/server/html
2020/07/25 07:21:55 Installed menu.html in C:\msys64\home\RajaS/server/html
2020/07/25 07:21:55 Installed stats.html in C:\msys64\home\RajaS/server/html

\end{lstlisting}

\end{landscape}

\section{Implementation}
 The following repository contains 2 projects - a cli template and a webserver template in the \textbf{go} language.
 
\paragraph{}
\fbox{%
    \parbox{6in}{%
    Implementation in \textbf{go} : \url{https://gitlab.com/projtemplates/go.git}
    }%
}

\end{document}