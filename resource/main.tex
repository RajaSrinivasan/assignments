% Credits are indicated where needed. The general idea is based on a template by Vel (vel@LaTeXTemplates.com) and Frits Wenneker.

\documentclass[12pt, a4paper]{article} % General settings in the beginning (defines the document class of your paper)
% 11pt = is the font size
% A4 is the paper size
% “article” is your document class

%----------------------------------------------------------------------------------------
%	Packages
%----------------------------------------------------------------------------------------

% Necessary
\usepackage[german,english]{babel} % English and German language 
\usepackage{booktabs} % Horizontal rules in tables 
% For generating tables, use “LaTeX” online generator (https://www.tablesgenerator.com)
\usepackage{comment} % Necessary to comment several paragraphs at once
\usepackage[utf8]{inputenc} % Required for international characters
\usepackage[T1]{fontenc} % Required for output font encoding for international characters
\usepackage{listings}

\lstset{basicstyle=\linespread{0.8}\ttfamily\footnotesize,
    xleftmargin=0.7cm,
    frame=tlbr, framesep=0.2cm, framerule=0pt,
}
\lstset{frame=single}

% Might be helpful
\usepackage{amsmath,amsfonts,amsthm} % Math packages which might be useful for equations
\usepackage{tikz} % For tikz figures (to draw arrow diagrams, see a guide how to use them)
\usepackage{tikz-cd}
\usetikzlibrary{positioning,arrows} % Adding libraries for arrows
\usetikzlibrary{decorations.pathreplacing} % Adding libraries for decorations and paths
\usepackage{tikzsymbols} % For amazing symbols ;) https://mirror.hmc.edu/ctan/graphics/pgf/contrib/tikzsymbols/tikzsymbols.pdf 
\usepackage{blindtext} % To add some blind text in your paper

\usepackage{url}
\usepackage{enumerate}
%---------------------------------------------------------------------------------
% Additional settings
%---------------------------------------------------------------------------------

%---------------------------------------------------------------------------------
% Define your margins
\usepackage{geometry} % Necessary package for defining margins

\geometry{
	top=2cm, % Defines top margin
	bottom=2cm, % Defines bottom margin
	left=2.2cm, % Defines left margin
	right=2.2cm, % Defines right margin
	includehead, % Includes space for a header
	%includefoot, % Includes space for a footer
	%showframe, % Uncomment if you want to show how it looks on the page 
}

\setlength{\parindent}{15pt} % Adjust to set you indent globally 

%---------------------------------------------------------------------------------
% Define your spacing
\usepackage{setspace} % Required for spacing
% Two options:
\linespread{1}
%\onehalfspacing % one-half-spacing linespread

%----------------------------------------------------------------------------------------
% Define your fonts
\usepackage[T1]{fontenc} % Output font encoding for international characters
\usepackage[utf8]{inputenc} % Required for inputting international characters

\usepackage{XCharter} % Use the XCharter font


%---------------------------------------------------------------------------------
% Define your headers and footers

\usepackage{fancyhdr} % Package is needed to define header and footer
\pagestyle{fancy} % Allows you to customize the headers and footers

%\renewcommand{\sectionmark}[1]{\markboth{#1}{}} % Removes the section number from the header when \leftmark is used

% Headers
\lhead{} % Define left header
\chead{\textit{}} % Define center header - e.g. add your paper title
\rhead{} % Define right header

% Footers
\lfoot{resource v0.1} % Define left footer
\rfoot{\footnotesize Page \thepage} % Define center footer
\cfoot{ } % Define right footer

%---------------------------------------------------------------------------------
%	Add information on bibliography
\usepackage{natbib} % Use natbib for citing
\usepackage{har2nat} % Allows to use harvard package with natbib https://mirror.reismil.ch/CTAN/macros/latex/contrib/har2nat/har2nat.pdf

% For citing with natbib, you may want to use this reference sheet: 
% http://merkel.texture.rocks/Latex/natbib.php

%---------------------------------------------------------------------------------
% Add field for signature (Reference: https://tex.stackexchange.com/questions/35942/how-to-create-a-signature-date-page)
\newcommand{\signature}[2][5cm]{%
  \begin{tabular}{@{}p{#1}@{}}
    #2 \\[2\normalbaselineskip] \hrule \\[0pt]
    {\small \textit{Signature}} \\[2\normalbaselineskip] \hrule \\[0pt]
    {\small \textit{Place, Date}}
  \end{tabular}
}
%---------------------------------------------------------------------------------
%	General information
%---------------------------------------------------------------------------------
\title{Resource support} % Adds your title
\author{Srini, rs@toprllc.com}
\date{\small \today} % Adds the current date to your “cover” page; leave empty if you do not want to add a date


%---------------------------------------------------------------------------------
%	Define what’s in your document
%---------------------------------------------------------------------------------

\begin{document}
%\newfontfamily\listingsfont[Scale=.7]{Menlo}

% If you want a cover page, uncomment "%---------------------------------------------------------------------------------
% Cover page
%---------------------------------------------------------------------------------

% Here are more templates for other cover pages: https://www.latextemplates.com/cat/title-pages

% This example is based on this cover page example: https://www.latextemplates.com/template/academic-title-page

\begin{titlepage} % Starts new environment where the page number is not displayed and the count starts at 1 for the next page

%------------------------------------------------
%	Institutional information
%------------------------------------------------
	
\begin{minipage}{0.4\textwidth} % Begins new environment (like a text box)
    \begin{flushleft} % Sets environment on the left side of the paper
    \large
    University of XX\\ % Add your institution
    Chair of Political Science IV\\ % Add the chair
    Fall 2018\\ % Add term
    COURSE TITLE\\ % Add course title
    Supervisor: NAME % Add instructor/supervisor name 
    \end{flushleft}
\end{minipage}
	
\vspace*{2in} % Adds some space in-between
	
\center % Centre everything on the page

%------------------------------------------------
%	Main part
%------------------------------------------------
	
{\huge\bfseries TITLE OF YOUR PAPER}\\[0.4cm] % Add your paper title 
{\large\today}\\[0.4cm] % Add date (current day)
FIRSTNAME LASTNAME % Add your name
	
\vfill % Adds additional space

%------------------------------------------------
%	General information about the author
%------------------------------------------------

\vfill % Adds additional space

Your contact info \\ % Add your contact info
Your Program \\ % Add info about your program
Semester you are enrolled \\ % Add info about your semester

\vfill % Adds additional space

%------------------------------------------------
%	Word count
%------------------------------------------------

\vfill % Adds additional space
	
Word count: XXXX % To indicate the word count
% How to check words in a LaTeX document: https://www.overleaf.com/help/85-is-there-a-way-to-run-a-word-count-that-doesnt-include-latex-commands
	

	
\end{titlepage}" and uncomment "\begin{comment}" and "\end{comment}" to comment the following lines
%%---------------------------------------------------------------------------------
% Cover page
%---------------------------------------------------------------------------------

% Here are more templates for other cover pages: https://www.latextemplates.com/cat/title-pages

% This example is based on this cover page example: https://www.latextemplates.com/template/academic-title-page

\begin{titlepage} % Starts new environment where the page number is not displayed and the count starts at 1 for the next page

%------------------------------------------------
%	Institutional information
%------------------------------------------------
	
\begin{minipage}{0.4\textwidth} % Begins new environment (like a text box)
    \begin{flushleft} % Sets environment on the left side of the paper
    \large
    University of XX\\ % Add your institution
    Chair of Political Science IV\\ % Add the chair
    Fall 2018\\ % Add term
    COURSE TITLE\\ % Add course title
    Supervisor: NAME % Add instructor/supervisor name 
    \end{flushleft}
\end{minipage}
	
\vspace*{2in} % Adds some space in-between
	
\center % Centre everything on the page

%------------------------------------------------
%	Main part
%------------------------------------------------
	
{\huge\bfseries TITLE OF YOUR PAPER}\\[0.4cm] % Add your paper title 
{\large\today}\\[0.4cm] % Add date (current day)
FIRSTNAME LASTNAME % Add your name
	
\vfill % Adds additional space

%------------------------------------------------
%	General information about the author
%------------------------------------------------

\vfill % Adds additional space

Your contact info \\ % Add your contact info
Your Program \\ % Add info about your program
Semester you are enrolled \\ % Add info about your semester

\vfill % Adds additional space

%------------------------------------------------
%	Word count
%------------------------------------------------

\vfill % Adds additional space
	
Word count: XXXX % To indicate the word count
% How to check words in a LaTeX document: https://www.overleaf.com/help/85-is-there-a-way-to-run-a-word-count-that-doesnt-include-latex-commands
	

	
\end{titlepage}

%\begin{comment}
\maketitle % Print your title, author name and date; comment if you want a cover page 
%\end{comment}

%----------------------------------------------------------------------------------------
% Introduction
%----------------------------------------------------------------------------------------
\setcounter{page}{1} % Sets counter of page to 1

\section{Introduction} % Add a section title

This projectlet enables applications to include reources such as graphics files, audio files in the
executable for distribution; instead of as individual files. This is achieved by converting the files into source code to be compiled for inclusion in the binaries. At runtime, the application can retrieve the resource with a runtime library.

\begin{itemize}
    \item Generated sources can be in one of several languages : \textbf{C, Ada, go}.
    \item A hash of the file is generated and included in the binary for runtime verification. This will protect the binary from corruption and/or tampering.
    \item Project configuration is provided in \textbf{json} format.
    \item The utility can implement a simplistic encryption of the data before generating the output. This is to enable incorporation of API keys and the like into the binary.The goal is to make it a little harder to extract the secrets. 
\end{itemize}

\subsection{Learning Objectives}

\begin{itemize}
    \item Process json formatted files. \textbf{json} is ubiquitous as the vehicle for data exchange. Most programming languages support json processing without each of us having to develop it from scratch. Choosing an appropriate library is part of this exercise.
    \item Polymorphism - Given a configuration in json, this projectlet converts files into source code compilable for inclusion in an executable. Depending on the user the preference may be for a variety of programming languages. Your programming language might enable this in some way e.g. \textbf{base class} and \textbf{derived} classes in \textbf{C++} while for \textbf{go} the technique might be built on the concept of \textbf{interface}s.
    \item In order to protect against corruption of the data even if included in a binary, we could several techniques. A simple way might be to calculate a checksum or hash and verify the value at runtime before using the data. Here again, an appropriate library has to be chosen instead of implementing the algorithm.
    \item Unit testing is an important aspect of software engineering. Though there seem to be numerous frameworks for unit testing, it does not have to be too fanciful. The project can be divided into smaller modules e.g. command line processing, loading the configuration data from the file and the actual conversion of files into resources; unit tests can be developed for each module.
    \item Exploration of a simplistic approach to data encryption/decryption.
\end{itemize}

\subsection{Configuration}

A sample configuration for a project file is listed below.

\begin{lstlisting}[caption=Example config.json]
{
    "resources" : [
        {
            "name" : "RESOURCE1" ,
            "filename" : "res1.png"
        } ,
        {
            "name" : "RESOURCE2" ,
            "filename" : "source.h"
        }
    ]
}

\end{lstlisting}

\section{Implementation}

An example in \textbf{C++} is developed using the following external libraries:
\paragraph{Libraries}
\begin{itemize}
    \item JSON processing is achieved using the \textbf{libjansson} (\url{https://digip.org/jansson/}). This library is a stable library in \textbf{C} - quite simple to integrate.
    \item Cryptographic hash of files is generated using the MD5 algorithm. The library GTK3 from the gnome project (\url{https://developer.gnome.org/references}) is used for this purpose.
\end{itemize}

\paragraph{Encryption}
The fundamental building block used is the \textbf{xor} algorithm. In order to keep the password management simple, the creation date of the output structure is used as the key for this algorithm. The input is processed in blocks of a configurable size and the timestamp is repeated as many times as needed to fill up the encryption key of size equal to this block length.

Exactly identical algorithm is performed by the runtime to decrypt the data. The required parameters namely the creation date and the blocklength are to be conveyed to the runtime - which is added to the output files - a header file in the case of C and a package spec in the case of Ada.

\paragraph{}
\fbox{%
    \parbox{40em}{%
    Implementation in \textbf{C++} : \url{https://gitlab.com/cpp8/bindata.git}
    }%
}

\subsection{Potential enhancements}
\begin{itemize}
    \item Text files can be potentially decoded from the hexadecimal conversions. An alternate representation might make it a little harder.
    \item Some files such as text files may benefit from compression. The compressed data can become the compilable source. 
\end{itemize}
\end{document}